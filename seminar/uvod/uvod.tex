\section{Uvod}

U današnjem digitalno vođenom poslovnom okruženju, sposobnost efikasnog prikupljanja, obrade i analize velikih količina podataka postala je ključni čimbenik uspjeha za organizacije u gotovo svim industrijskim granama. Automobilska industrija, kao jedna od najkompleksnijih i najkonkurentnijih grana gospodarstva, posebno se oslanja na napredne tehnologije skladištenja i analize podataka kako bi razumjela tržišne trendove, potrebe kupaca i operacijske učinkovitosti \cite{Silva2021}.

Skladišta podataka (eng. \textit{data warehouses}) predstavljaju temelj modernih sustava za podršku odlučivanju, omogućujući integraciju različitih izvora podataka u jedinstvenu, koherentnu strukturu optimiziranu za analitičke potrebe \cite{Garani2019}. U kontekstu automobilske industrije, ovakvi sustavi omogućuju analizu složenih odnosa između cijena vozila, karakteristika proizvođača, tržišnih segmenata i regionalnih specifičnosti.

Ovaj rad predstavlja sveobuhvatan pristup razvoju sustava za skladištenje i rudarenje podataka primjenjenog na analizu automobilskog tržišta. Kroz razvoj kompletnog ETL (Extract, Transform, Load) procesa, projekt demonstrira transformaciju sirovih podataka o automobilima u strukturirani dimenzijski model prilagođen OLAP (Online Analytical Processing) analizama. Korištenjem skupa podataka koji sadrži preko 97.000 zapisa o vozilima različitih proizvođača, modela i karakteristika, razvijen je sustav koji omogućuje dubinsku analizu tržišnih trendova i poslovnih uvida.

Glavni cilj ovog projekta je ilustracija praktične primjene teorijskih koncepata skladišta podataka kroz razvoj funcionalnog sustava koji može poslužiti organizacijama poput autoklubova, analitičkih kuća ili samim proizvođačima automobila u donošenju informiranih poslovnih odluka \cite{Nima2018}. Projekt obuhvaća sve ključne faze razvoja sustava - od eksploratorne analize početnih podataka, preko dizajna relacijskog i dimenzijskog modela, do implementacije ETL procesa i prijedloga OLAP analiza.

Struktura rada prati logični tijek razvoja sustava, počevši od analize i pripreme početnog skupa podataka, preko stvaranja normaliziranog relacijskog modela, do konačne implementacije zvjezdastog modela optimiziranog za analitičke potrebe. Svaki korak popraćen je detaljnim objašnjenjima projektnih odluka i praktičnih implementacijskih izazova, čineći ovaj rad korisnim resursom za razumijevanje kompleksnosti razvoja realnih sustava za skladištenje podataka.

Kroz ovaj projekt, nastoji se pokazati kako tehnologije poput Apache Spark-a, MySQL-a i Tableau-a mogu biti integrirane u koherentan sustav koji omogućuje ne samo tehnički ispravan rad, već i stvaranje dodane vrijednosti kroz kvalitetne poslovne uvide \cite{Leka2025}.


