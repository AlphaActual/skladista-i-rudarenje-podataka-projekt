\documentclass[a4paper,12pt]{article}
\usepackage[utf8]{inputenc}
\usepackage[T1]{fontenc}
\usepackage{lmodern}
\usepackage{geometry}
\geometry{
 a4paper,
 left=1.25in,
 right=1.25in,
 top=1in,
 }
\usepackage[croatian,english]{babel}    %za hrvatske naslove
\usepackage[nottoc]{tocbibind}  %za pravilan table of content
\usepackage{graphicx}   %za dodavanje slika
\usepackage{amsmath}    %za matematičke forumele
\usepackage{subcaption} %za dvije slike u jednoj
\usepackage{booktabs}   %za bolje tablice
\usepackage{multirow}   %za multirow tablice
\usepackage{cite}
\usepackage{url}       %za URL linkove u bibliografiji
\usepackage{float}
\usepackage{hyperref}  %za hiperlinkove - mora biti zadnji
\usepackage{listings}
\usepackage{xcolor}
\usepackage{listingsutf8}

% Configure listings for Python
\lstset{
    language=Python,
    basicstyle=\ttfamily\small,
    keywordstyle=\color{purple!70!white},
    commentstyle=\color{green!60!black},
    stringstyle=\color{orange!70!black},
    numberstyle=\tiny\color{gray!70!white},
    showstringspaces=false,
    breaklines=true,
    frame=single,
    numbers=left,
    inputencoding=utf8,
    extendedchars=true
}

% Konfiguracija hyperref-a za boje i stil linkova
\hypersetup{
    colorlinks=true,        % linkovi su obojeni umjesto u okvirima
    linkcolor=black,         % boja unutarnjih linkova (TOC, reference)
    citecolor=black,        % boja citata
    urlcolor=blue,          % boja URL linkova
    filecolor=magenta,      % boja linkova na datoteke
    menucolor=red,          % boja linkova na meni
    breaklinks=true,        % dozvoljava lomljenje linkova preko redaka
    % bookmarks=true,       % stvara PDF bookmarks (default je true)
    % bookmarksopen=true,   % otvara bookmark panel
    % pdfstartview=FitH,    % kako se PDF otvara
    % pdftitle={Naslov rada},    % naslov PDF-a
    % pdfauthor={Vaše ime},      % autor PDF-a
    % pdfsubject={Tema rada},    % tema PDF-a
}

\begin{document}

\begin{center}
SVEUČILIŠTE JURJA DOBRILE U PULI 

FAKULTET INFORMATIKE

\vfill

\textbf{Tin Pritišanac}

\vspace{20mm} 

\textbf{Analiza tržišta automobila 1970.-2024.}

\vspace{5mm}
SEMINARSKI RAD

\vfill

%Upisat tocan mjesec i godinu
Pula, rujan, 2025. godine
\end{center}

\pagenumbering{gobble}
\clearpage
\newpage

\begin{center}
SVEUČILIŠTE JURJA DOBRILE U PULI 

FAKULTET INFORMATIKE

\vspace{45mm} 

\textbf{Tin Pritišanac}

\vspace{20mm} 

    \textbf{Analiza tržišta automobila 1970.-2024.}

\vspace{5mm}
SEMINARSKI RAD

\end{center}

\vspace{45mm}

\textbf{JMBAG: 0171256219, izvanredni student}

\textbf{Studijski smjer: Informatika}

\textbf{Kolegij: \textbf{Skladišta i rudarenje podataka} }

\textbf{Mentor: doc.dr.sc. Goran Oreški}

\vfill

\begin{center}

%Upisat tocan mjesec i godinu
Pula, rujan, 2025. godine

\end{center}
\pagenumbering{gobble}
\clearpage
\newpage

\begin{center}
    \begin{figure}[H]
        \centering
        \includegraphics[width=4cm]{slike/uni-logo.jpg}
    \end{figure}

    \vspace{45mm}

    IZJAVA O AKADEMSKOJ ČESTITOSTI

    \vspace{20mm}

    \begin{flushleft}
        Ja, dolje potpisan Tin Pritišanac, ovime izjavljujem da je ovaj seminarski rad rezultat isključivo
        mojega vlastitog rada, da se temelji na mojim istraživanjima te da se oslanja na objavljenu
        literaturu kao što to pokazuju korištene bilješke i bibliografija. Izjavljujem da niti jedan dio
        seminarskog rada nije napisan na nedozvoljen način, odnosno da je prepisan iz kojega
        necitiranog rada, te da ikoji dio rada krši bilo čija autorska prava. Izjavljujem, također, da
        nijedan dio rada nije iskorišten za koji drugi rad pri bilo kojoj drugoj visokoškolskoj,
        znanstvenoj ili radnoj ustanovi.
    \end{flushleft}

    \vspace{20mm}
    \begin{flushright}
        STUDENT~~~~~~~~~~~~~~~ \\
        \bigskip
        \bigskip
        \rule{4cm}{0.4pt}
    \end{flushright}

    \vspace{30mm}

    %Upisat tocan mjesec i godinu
    Pula, rujan, 2025. godine
\end{center}

\pagenumbering{gobble}
\clearpage
\newpage

%Resetirat margine
\restoregeometry

\clearpage
\newpage

\selectlanguage{croatian}
\tableofcontents
\pagenumbering{gobble}
\clearpage
\newpage

\pagenumbering{arabic}

\section{Uvod}

Ovo je samo test citata \cite{example}.

\newpage

\section{Projektni zadatak}

\subsection{Definicija problema i motivacija}

Automobilska industrija je jedna od najkompleksnijih gospodarskih grana koja generira ogromne količine podataka - od osnovnih karakteristika vozila, preko cijena i tržišnih trendova, do regionalnih specifičnosti i preferencija kupaca. Organizacije koje se bave analizom automobilskog tržišta, poput autoklubova, analitičkih kuća ili samih proizvođača, suočavaju se s izazovom pretvaranja ovih velikih količina podataka u korisne poslovne uvide.

Tradicionalni pristup analize podataka korištenjem jednostavnih baza podataka i osnovnih alata za izvještavanje često se pokazuje neadekvatnim za složene analitičke potrebe. Potreba za dubinskim analizama tržišnih trendova, usporednim analizama proizvođača, segmentacijom kupaca i predviđanjem buduće dinamike tržišta zahtijeva sofisticiraniji pristup \cite{Silva2021}.

Upravo tu se prepoznaje potreba za razvojem naprednog sustava skladištenja i analize podataka koji će omogućiti organizacijama da iz sirovih podataka izvuku maksimalnu vrijednost te donose informirane strateške odluke.

\subsection{Cilj i opis projektnog zadatka}

Glavni cilj ovog projekta je razvoj potpunog sustava za skladištenje i rudarenje podataka prilagođenog analizi automobilskog tržišta. Projekt obuhvaća kompletan tijek rada - od početnih sirovih podataka do konačnih analitičkih izvještaja koji mogu koristiti stvarnim organizacijama u njihovom poslovanju.

Konkretno, projekt ima za cilj:

\begin{itemize}
    \item \textbf{Dizajnirati i implementirati relacijski model podataka} koji odgovara strukturi realnih podataka o automobilima, uključujući sve važne entitete i njihove međusobne odnose
    \item \textbf{Razviti dimenzijski model (star schema)} optimiziran za OLAP analize, koji omogućuje efikasno izvršavanje složenih analitičkih upita
    \item \textbf{Implementirati ETL proces} koji transformira podatke iz relacijske strukture u dimenzijski model koristeći suvremene tehnologije poput Apache Spark-a
    \item \textbf{Kreirati sustav za OLAP analize} koji demonstrira praktičnu primjenu različitih analitičkih operacija (slice, dice, drill-down, roll-up, pivot)
    \item \textbf{Pokazati poslovnu vrijednost} kroz konkretne scenarije korištenja koji ilustriraju kako ovakav sustav može koristiti organizacijama u donošenju poslovnih odluka
\end{itemize}

\subsection{Opseg i ograničenja projekta}

Projekt se fokusira na analizu podataka o automobilima prodavanima na fiktivnom tržištu, koristeći skup od preko 97.000 zapisa koji obuhvaća vozila različitih proizvođača, modela i karakteristika. Ovaj skup podataka pruža reprezentativan uzorak koji omogućuje demonstraciju svih ključnih koncepata skladišta podataka.

Vremenski okvir podataka pokriva razdoblje od 1970. do 2024. godine, s naglaskom na zadnja dva desetljeća, što omogućuje analizu dugoročnih trendova i promjena na tržištu. Podaci uključuju ključne atribute poput cijene, godine proizvodnje, kilometraže, vrste goriva, veličine motora i drugih karakteristika relevantnih za tržišnu analizu.

Projekt se ograničava na demonstraciju tehničkih mogućnosti i metodologija, a ne pretendira na potpunu komercijalnu implementaciju. Fokus je na edukacijskim aspektima i ilustraciji najboljih praksi u razvoju sustava za skladištenje podataka.

\subsection{Očekivani rezultati i doprinosi}

Na završetku projekta očekuje se sljedeće:

\begin{enumerate}
    \item \textbf{Funkcionalno skladište podataka} s implementiranim relacijskim i dimenzijskim modelom podataka
    \item \textbf{Potpuno funkcionalan ETL proces} koji automatizira transformaciju podataka između različitih modela
    \item \textbf{Demonstracija OLAP mogućnosti} kroz konkretne analitičke scenarije i upite
    \item \textbf{Dokumentacija procesa} koja može služiti kao vodič za buduće slične projekte
    \item \textbf{Praktični uvidi u tržište automobila} koji ilustriraju poslovnu vrijednost analitičkih sustava
\end{enumerate}

Ovaj projekt predstavlja praktičnu demonstraciju kako teorijski koncepti skladišta podataka i business intelligence mogu biti primijenjeni u realnom kontekstu, pružajući studentima i praktičarima vrijedan uvid u izazove i mogućnosti moderne analize podataka \cite{Garani2019}.

\subsection{Metodologija rada}

Projekt slijedi strukturiran pristup razvoja sustava za skladištenje podataka, koji se može podijeliti u pet glavnih faza:

\begin{enumerate}
    \item \textbf{Eksploratorna analiza podataka} - detaljno istraživanje početnog skupa podataka radi razumijevanja strukture, kvalitete i potencijalnih izazova
    \item \textbf{Dizajn relacijskog modela} - stvaranje normalizirane strukture podataka koja odražava realne poslovne entitete i njihove odnose
    \item \textbf{Implementacija dimenzijskog modela} - razvoj star schema arhitekture optimizirane za analitičke potrebe
    \item \textbf{ETL proces} - implementacija sustava za ekstraktiranje, transformaciju i učitavanje podataka koristeći Apache Spark
    \item \textbf{OLAP analize} - demonstracija analitičkih mogućnosti kroz praktične scenarije kreiranja grafova i tablica u programu Tableau
\end{enumerate}

Svaka faza dokumentirana je s objašnjenjima projektnih odluka, tehničkih izazova i načina njihova rješavanja, čineći projekt korisnim resursom za razumijevanje praktičnih aspekata razvoja sustava za skladištenje podataka.

\newpage

\section{Odabir i analiza skupa podataka}

\subsection{Izvor podataka}
\subsection{Analiza podataka}
\subsection{Priprema podataka}
\newpage

\section{Relacijski model podataka}

Nakon uspješno provedene analize podataka, sljedeći korak u razvoju skladišta podataka predstavlja kreiranje relacijskog modela koji će osigurati strukturirano i normalizirano čuvanje podataka. Relacijski model omogućuje organizaciju podataka u tablice povezane preko stranih ključeva, što omogućava efikasno dohvaćanje i manipulaciju podataka \cite{Elmasri2017}.

\subsection{Analiza entiteta i atributa}

Temeljito razumijevanje strukture podataka o automobilima omogućilo je identifikaciju ključnih entiteta koji će činiti okosnicu relacijskog modela. Kroz analizu originalnog skupa podataka identificirani su sljedeći glavni entiteti:

\textbf{Osnovni entiteti:}
\begin{itemize}
    \item \textbf{Automobil} - središnji entitet koji sadrži osnovne karakteristike vozila
    \item \textbf{Proizvođač} - entitet koji predstavlja tvrtke koje proizvode automobile
    \item \textbf{Model} - specifični model automobila određenog proizvođača
    \item \textbf{Zemlja} - zemlja podrijetla proizvođača
    \item \textbf{Regija} - geografska regija kojoj pripada zemlja
\end{itemize}

\textbf{Klasifikacijski entiteti:}
\begin{itemize}
    \item \textbf{Tip mjenjača} - klasifikacija prema vrsti transmisije
    \item \textbf{Tip goriva} - kategorije goriva koje automobil koristi
    \item \textbf{Desetljeće} - vremenski period proizvodnje
    \item \textbf{Kategorija starosti} - klasifikacija prema godinama starosti
    \item \textbf{Kategorija kilometraže} - klasifikacija prema prijeđenim kilometrima
    \item \textbf{Klasa veličine motora} - kategorije prema volumenu motora
\end{itemize}

Ova kategorizacija omogućuje stvaranje normaliziranog modela koji minimizira redundanciju podataka i omogućuje efikasne upite.

\subsection{Konceptualni model podataka}

Konceptualni model predstavlja visokorazinsku apstrakciju poslovnih zahtjeva bez ulaska u tehnička ograničenja. Za potrebe ovog projekta definiran je model koji odražava prirodne veze između entiteta u domeni trgovine automobilima.

Ključne veze u modelu uključuju:
\begin{itemize}
    \item Svaki automobil pripada određenom modelu (1:N)
    \item Svaki model proizvodi točno jedan proizvođač (1:N)
    \item Svaki proizvođač dolazi iz jedne zemlje (1:N)
    \item Svaka zemlja pripada jednoj regiji (1:N)
    \item Automobil ima jednu kategoriju za svaki klasifikacijski atribut (1:N)
\end{itemize}

\begin{figure}[H]
    \centering
    \includegraphics[width=0.9\textwidth]{slike/relational_model/er-dijagram.png}
    \caption{Konceptualni Entity-Relationship dijagram}
    \label{fig:er-dijagram}
\end{figure}


ER dijagram prikazuje kompletan konceptualni model s entitetima, atributima i vezama između njih. Dijagram je kreiran koristeći standardnu notaciju koja jasno označava kardinalnosti i ograničenja. Na dijagramu je jasno vidljiva hijerarhijska struktura od regije do modela automobila, kao i klasifikacijski entiteti koji omogućuju analitičko izvještavanje. Svaki entitet sadrži primarni ključ i relevantne atribute koji podržavaju analitičke zahtjeve.

\subsection{Implementacija relacijskog modela}

Logički model podataka implementiran je koristeći MySQL bazu podataka, pri čemu je osobita pozornost posvećena normalnim formama i referencijalnim ograničenjima. Model slijedi treću normalnu formu (3NF) što osigurava minimizaciju redundancije podataka.

\begin{figure}[H]
    \centering
    \includegraphics[width=1.0\textwidth]{slike/relational_model/relacijski-model.png}
    \caption{Implementirani relacijski model u MySQL bazi podataka}
    \label{fig:relacijski-model}
\end{figure}

Implementacija uključuje:
\begin{itemize}
    \item \textbf{Referencijalnu cjelovitost} - svi strani ključevi osiguravaju postojanje povezanih zapisa
    \item \textbf{Jedinstvene ograničenja} - sprječavanje dupliciranja entiteta
    \item \textbf{NOT NULL ograničenja} - osiguravanje kompletnosti kritičnih atributa
    \item \textbf{Auto-increment primarni ključevi} - efikasno upravljanje identifikatorima
\end{itemize}

\subsection{Automatizacija kreiranja i popunjavanja baze}

Za potrebe reproducibilnosti i održivosti projekta razvijene su Python skripte koje automatiziraju proces kreiranja i popunjavanja baze podataka. Glavna skripta koristi SQLAlchemy ORM za definiranje sheme baze.

\begin{lstlisting}[language=Python, caption={Definiranje ORM modela za glavni entitet automobila}]
class Car(Base):
    __tablename__ = 'car'
    car_id = Column(Integer, primary_key=True, autoincrement=True)
    year = Column(Integer, nullable=False)
    price = Column(Integer, nullable=False)
    mileage = Column(Integer, nullable=False)
    tax = Column(Integer, nullable=False)
    mpg = Column(Float, nullable=False)
    engineSize = Column(Float, nullable=False)
    age = Column(Integer, nullable=False)
    
    # Strani kljucevi
    model_id = Column(Integer, ForeignKey('model.model_id'))
    transmission_id = Column(Integer, ForeignKey('transmission_type.transmission_id'))
    fuel_id = Column(Integer, ForeignKey('fuel_type.fuel_id'))
    # ... ostali strani kljucevi
\end{lstlisting}

Proces popunjavanja baze provodi se u kontroliranim fazama:
\begin{enumerate}
    \item \textbf{Kreiranje osnovnih entiteta} - regije, zemlje, proizvođači
    \item \textbf{Kreiranje klasifikacijskih tabela} - tipovi goriva, mjenjača, kategorije
    \item \textbf{Kreiranje modela} - povezivanje s proizvođačima
    \item \textbf{Umetanje automobila} - povezivanje sa svim referentnim entitetima
\end{enumerate}

\subsection{Validacija podataka i testiranje}

Za osiguravanje ispravnosti procesa migracije podataka u zasebnoj skripti implementiran je sveobuhvatan sustav testiranja koji uspoređuje originalne CSV podatke s podacima u bazi.

\begin{lstlisting}[language=Python, caption={SQL upit za rekonstrukciju originalnih podataka}]
query = """
SELECT mfg.name as 'manufacturer', mdl.name as 'model'
, c.year, c.price, c.mileage, c.tax, c.mpg, c.engineSize
, t.type as 'transmission', f.type as 'fuelType'
, d.decade, cnt.name as 'country', r.name as 'region'
, mc.category as 'mileageCategory'
, esc.size_class as 'engineSizeClass'
, c.age, ac.category as 'ageCategory'
FROM car c
JOIN model mdl ON c.model_id = mdl.model_id
JOIN manufacturer mfg ON mdl.manufacturer_id = mfg.manufacturer_id
-- ... ostali JOIN-ovi
ORDER BY c.car_id ASC
"""
\end{lstlisting}

Test provjerava:
\begin{itemize}
    \item \textbf{Integritet strukture} - postojanje svih stupaca u bazi
    \item \textbf{Kompletnost podataka} - broj zapisa u bazi odgovara CSV datoteci
    \item \textbf{Ispravnost sadržaja} - vrijednosti u bazi identične su originalnim podacima
    \item \textbf{Referencijalnu konzistentnost} - svi strani ključevi ispravno povezani
\end{itemize}

\newpage

\section{Dimenzijski model podataka}

\subsection{Izrada star scheme}
\subsection{Kreiranje dimenzijskih tablica}
\subsection{Kreiranje tablice činjenica}
\newpage

\section{ETL proces}

\subsection{Izvlačenje podataka}
\subsection{Transformacija podataka}
\subsection{Popunjavanje skladišta podataka}
\newpage

\section{OLAP analiza}

\subsection{Definiranje prikaza podataka}
\subsection{Vizualizacija podataka u Tableau}
\subsubsection{Graf 1 TODO}
\newpage

\input{zaključak/zaključak}
\newpage

\bibliographystyle{unsrt}
\bibliography{literatura}
\newpage

%Automatski generira listu svih slika
\listoffigures
\newpage

%Automatski generira listu svih tablica
\listoftables
\newpage


\end{document}
